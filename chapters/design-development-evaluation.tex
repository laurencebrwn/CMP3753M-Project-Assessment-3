\section{Project Requirements}
\subsection{Software Requirements}
In order to reproduce the artefact, software requirements must be installed and hardware requirements must be met as prerequisites to the project. While some software, like the development environments are optional and are not needed to train the models, some modification to the code might be needed to run without them.

\phantomsection
\subsubsection{Python}
The entirety of the projects code was written in Python and such is a hard requirement to run the projects artefact. It is recommended to use Python 3.7 or later for TensorFlow support.

\subsubsection{TensorFlow and Keras}
The TensorFlow and Keras are used in the entirety of this project and such is a hard requirement to run the artefact. It is recommended to use TensorFlow 2.0 and Keras 2.0 or later, but with some code modifications, TensorFlow 1.0+ and Keras 1.0+ can also be used.

\subsubsection{TensorBoard}
In this project, TensorBoard 2.8.0 was used and is a hard requirement to run the artefact. As with TensorFlow, it is recommended to use TensorBoard 2.0 or later, but with some code modifications, TensorBoard 1.0+ can also be used.

\subsubsection{Scikit-Learn}
In this project, Scikit-Learn 1.0.2 was used and is a hard requirement to run the artefact.

\subsubsection{pandas}
In this project, pandas 1.3.5 was used and is a hard requirement to run the artefact.

\subsubsection{NumPy}
In this project, NumPy 1.21.6 was used and is a hard requirement to run the artefact.

\subsubsection{Google Cloud Platform - Development Environment}
Google Cloud Platform and its subsidiary solutions are not a requirement to run the artefact, however can be used to run model training simultaneously.

\subsubsection{Google Colab - Development Environment}
Google Colab is not a requirement to run the artefact, however a Jupyter notebook solution, like this, must be used.

\subsubsection{Project Code}
The project code is hosted on GitHub, and can be accessed and downloaded through the following link: \url{https://github.com/laurencebrwn/ml-covid19-eval}. The results and individual models notebook files are all found here.

\subsubsection{COVIDx-CXR Dataset}
The dataset used to train the image classification models is a hard requirement for this project. The full dataset is hosted on kaggle, and can be found here: \url{https://github.com/lindawangg/COVID-Net/blob/master/docs/COVIDx.md}. The images should be obtained from the "COVIDx CXR-2 Kaggle Dataset" link (\url{https://www.kaggle.com/datasets/andyczhao/covidx-cxr2} detailed on the previously linked page. The labels for both binary ("train\_COVIDx9B.txt" and "test\_COVIDx9B.txt") and multi-label ("train\_COVIDx9A.txt" and "test\_COVIDx9A.txt") classification, can be downloaded from the link \url{https://github.com/lindawangg/COVID-Net/tree/master/labels} detailed on the previously linked page.

\subsection{Hardware Requirements}
While TensorFlow can run on a large number of devices, there are some hardware limitations, especially if a Graphical Processing Unit (GPU) is used to accelerate performance. TensorFlow requires a CPU capable of AVX instructions, which most modern Computational Processing Unit's (CPU) support \citep{InstallT17:online}. If GPU acceleration is required and the device used has an NVIDIA GPU, then it must support compute unified device architecture (CUDA), a "parallel computing platform" which can "dramatically speed up computing applications by harnessing the power of GPUs" \citep{CUDAZone2:online}. If the device has a GPU that does not support CUDA, but does support DirectX 12 (e.g. an AMD GPU manufactured in the last several years), then Microsoft's DirectML can be used enable GPU support for TensorFlow \citep{Introduc93:online}. Precautions must be taken if DirectML is used however, as it has differing version requirements of Python, TensorFlow and NumPy.

\section{Design}

\section{Building and Programming}

\section{Operation of Artefact}

\section{Results}
