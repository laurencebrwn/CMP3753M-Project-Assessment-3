\section{Project Requirements}
\subsection{Software Requirements}
In order to reproduce the artefact, software requirements must be installed and hardware requirements must be met as prerequisites to the project. While some software, like the development environments are optional and are not needed to train the models, some modification to the code might be needed to run without them.

\phantomsection
\subsubsection{Python}
The entirety of the projects code was written in Python and such is a hard requirement to run the projects artefact. It is recommended to use Python 3.7 or later for TensorFlow support.

\subsubsection{TensorFlow and Keras}
The TensorFlow and Keras are used in the entirety of this project and such is a hard requirement to run the artefact. It is recommended to use TensorFlow 2.0 and Keras 2.0 or later, but with some code modifications, TensorFlow 1.0+ and Keras 1.0+ can also be used.

\subsubsection{TensorBoard}
In this project, TensorBoard 2.8.0 was used and is a hard requirement to run the artefact. As with TensorFlow, it is recommended to use TensorBoard 2.0 or later, but with some code modifications, TensorBoard 1.0+ can also be used.

\subsubsection{Scikit-Learn}
In this project, Scikit-Learn 1.0.2 was used and is a hard requirement to run the artefact.

\subsubsection{pandas}
In this project, pandas 1.3.5 was used and is a hard requirement to run the artefact.

\subsubsection{NumPy}
In this project, NumPy 1.21.6 was used and is a hard requirement to run the artefact.

\subsubsection{Google Cloud Platform - Development Environment}
Google Cloud Platform and its subsidiary solutions are not a requirement to run the artefact, however can be used to run model training simultaneously.

\subsubsection{Google Colab - Development Environment}
Google Colab is not a requirement to run the artefact, however a Jupyter notebook solution, like this, must be used.

\subsubsection{Project Code}
The project code is hosted on GitHub, and can be accessed and downloaded through the following link: \url{https://github.com/laurencebrwn/ml-covid19-eval}. The results and individual models notebook files are all found here.

\subsubsection{COVIDx-CXR Dataset}
The dataset used to train the image classification models is a hard requirement for this project. The full dataset is hosted on kaggle, and can be found here: \url{https://github.com/lindawangg/COVID-Net/blob/master/docs/COVIDx.md}. The images should be obtained from the "COVIDx CXR-2 Kaggle Dataset" link (\url{https://www.kaggle.com/datasets/andyczhao/covidx-cxr2}) detailed on the previously linked page. The labels for both binary ("train\_COVIDx9B.txt" and "test\_COVIDx9B.txt") and multi-label ("train\_COVIDx9A.txt" and "test\_COVIDx9A.txt") classification, can be downloaded from the link (\url{https://github.com/lindawangg/COVID-Net/tree/master/labels}) detailed on the previously linked page.

\subsection{Hardware Requirements}
While TensorFlow can run on a large number of devices, there are some hardware limitations, especially if a Graphical Processing Unit (GPU) is used to accelerate performance. TensorFlow requires a CPU capable of AVX instructions, which most modern Computational Processing Unit's (CPU) support \citep{InstallT17:online}. If GPU acceleration is required and the device used has an NVIDIA GPU, then it must support compute unified device architecture (CUDA), a "parallel computing platform" which can "dramatically speed up computing applications by harnessing the power of GPUs" \citep{CUDAZone2:online}. If the device has a GPU that does not support CUDA, but does support DirectX 12 (e.g. an AMD GPU manufactured in the last several years), then Microsoft's DirectML can be used enable GPU support for TensorFlow \citep{Introduc93:online}. Precautions must be taken if DirectML is used however, as it has differing version requirements of Python, TensorFlow and NumPy.

\section{Design}
In order to design the software that would train and evaluate each of the selected models, identified in the literature review, prompts were taken from the methodology to devise a structure that would form each models notebook. While there are a few variations for each model a common format was followed for all models:

\begin{enumerate}
    \item Set up the environment.
    \begin{enumerate}
        \item Define the model variables, like the image size, batch size, dataset size, number of K folds and maximum amount of epochs.
        \item Import the required libraries, like TensorFlow, Keras, Scikit-Learn, etc.
        \item Define the hyper-parameters of the model (or list of hyper-parameters if performing tuning).
    \end{enumerate}
    \item Prepare the dataset.
    \begin{enumerate}
        \item Read in the label files for both test and train.
        \item Down-sample each class to match the total dataset size and to ensure all classes are equally represented.
        \item Shuffle the data.
        \item Pre-process the X-Ray images (and perform image transformations on the training dataset).
        \item Split the training data into folds for stratified K fold cross validation.
    \end{enumerate}
    \item Create the machine learning model.
    \begin{enumerate}
        \item Create the Keras sequential model.
        \item If the model is using transfer learning (Xception, ResNet and DenseNet), fetch the base model from Keras' application library.
        \item Define the model layers in the sequential model.
        \item Set any hyper-parameters, like the optimiser, learning rate etc.
        \item Compile the model.
    \end{enumerate}
    \item Begin cross validation training.
    \begin{enumerate}
        \item Iterate through hyper-parameter variations.
        \item Iterate through stratified K folds.
        \item Train the model.
        \item Record model logs.
        \item Save the model weights.
    \end{enumerate}
    \item Evaluate the model.
    \begin{enumerate}
        \item Iterate through the trained models.
        \item Load model weights.
        \item Predict the labels on the test dataset.
        \item Record the models key metrics, described in the methodology.
        \item Save the models metrics and performance
    \end{enumerate}
\end{enumerate}

\section{Developing the Artefact}
Developing the Python notebooks for each model followed a similar flow to that listed in the design section. Firstly the Environment set up was developed, followed by dataset preparation and pre-processing. Where the development flow differed was that the cross validation training and evaluation sections were developed before the model itself. This meant a skeletal structure consisting of everything except for the model itself could be replicated, once for each model. This section will detail the development via a run through of the code itself, split into the sections listed in the flow above.

\subsection{Environment Set Up}


\section{Operation of Artefact}

\section{Results}
