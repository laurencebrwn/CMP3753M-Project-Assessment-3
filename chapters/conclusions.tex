This study set out to analyse of the effectiveness of multiple machine learning algorithms to establish the best suited at diagnosing COVID-19. Throughout numerous trials and comparisons almost 100 models of varying configurations were tested. From the analysis of the results in the previous section the model best suited at diagnosing COVID-19 is the Xception network which made use of transfer learning from the ImageNet weights. Modifications to the network that proved the strongest were the addition of a GlobalAveragePooling2D layer that received Xception’s output, followed by a Dense layer with a rectified linear activation function (ReLU), followed by a BatchNormalization layer and a Dropout layer. The hyper-parameters that functioned best with this modified network were 128 Dense units, a dropout of 0.1, the Adam optimiser and Learning rate of 0.001. For the sake of this conclusion this specific model and configuration will be referred to as XceptionCov19. The full model diagram of XceptionCov19 can be seen in \autoref{fig:xceptioncov19}.

Comparing these results to other studies identified within the literature review, the best model identified in this study has differing results. Comparing firstly with \cite{zhang2020covid19xraynet} COVID19XrayNet model which produced an accuracy of 0.9192 when trained on a multi-label classification dataset, the XceptionCov19 model built in this study outperforms it by a large margin. When XceptionCov19 was trained on a small multi-label classification dataset it produced an accuracy result of 0.9335, and a result of 0.9530 when trained on a large multi-label classification dataset. Using metrics from the large multi-label classification dataset training, XceptionCov19 improves on COVID19XrayNet's performance by 0.0338, a 3.7\% increase in accuracy. COVID19XrayNet only trained on two databases that combined to a total of 6,195 images however, even when XceptionCov19 trained on a smaller dataset of 1,500 images it still produced a better result.

Comparing the results of XceptionCov19 against those of \cite{fitriasari2021improvement} Xception and ResNet inspired model, the multi-label classification results are closer. When \cite{fitriasari2021improvement} model trained on a total of 1,398 images of 4 classes, compared to XceptionCov19's 3 multi-label classes, it produced accuracy results of 0.9341, a marginal improvement over XceptionCov19's 0.9335 when trained on the small multi-label classification dataset, but not as strong as XceptionCov19's accuracy of 0.9530 when trained on a large multi-label classification dataset.

The \cite{bressem2020comparing} study compared many different models, much like this study, and its best performing model was DenseNet201. This model produced a pooled AU-ROC score of 0.998 when training on a 3 class multi-label dataset of 46,754 images. While the study does not provide accuracy results to compare with, AU-ROC scores were calculated separately for this comparison. The AU-ROC graphs for XceptionCov19 can be seen in \autoref{fig:au-roc-all}. XceptionCov19's mean AU-ROC score across its five folds is 0.962 when performing multi-label classification on the large dataset. While \cite{bressem2020comparing} DensNet201 did perform better, the model was also pre-trained on the CheXpert dataset (a collection of over 200,000 chest X-Ray images), which is likely to have improved DenseNet201's performance significantly.

With these findings it is clear that XceptionCov19 is at the forefront of COVID-19 classification, sharing similar and in some cases improved results over other state-of-the art models currently developed. XceptionCov19's accuracy percentage is 97.80\%, this is far above the 71.4\% accuracy of doctors' clinical vignettes diagnoses tests \citep{richens2020improving}. XceptionCov19's AU-ROC score is also higher than those of PCR tests, with 0.962 versus 0.879 respectively \citep{mardani2020laboratory}.

The performance of XceptionCov19 could be further improved in the future when larger and more diverse datasets become available. If computing power availability and time was not an issue more improvements could also be made if XceptionCov19 was pre-trained on a medicinal dataset, such as CheXpert. 

To conclude the XceptionCov19 model is among the best in the current space of COVID-19 classification models at diagnosing COVID-19 cases. It should be able to successfully support a doctor's diagnoses by providing a second opinion if implemented as an automatic diagnoses tool within hospitals.