From the analysis of the results in the previous section the model best suited at diagnosing COVID-19 is the Xception network. Modifications to the network that proved the strongest were the addition of a GlobalAveragePooling2D layer that received Xception’s output, followed by a Dense layer with a rectified linear activation function (ReLU), followed by a BatchNormalization layer and a Dropout layer. The hyper-parameters that functioned best with this modified network were 128 Dense units, a dropout of 0.1, the Adam optimiser and Learning rate of 0.001. For the sake of this conclusion this specific model and configuration will be refered to as XceptionCov19.

Comparing these results to other studies identified within the literature review, the best model identified in this study has differing results. Comparing firstly with \cite{zhang2020covid19xraynet} COVID19XrayNet model which produced an accuracy of 0.9192 when trained on a multi-label classification dataset, the XceptionCov19 model built in this study outperforms it by a large margin. When XceptionCov19 was trained on a small multi-label classification dataset it produced an accuracy result of 0.9335, and a result of 0.9530 when trained on a large multi-label classification dataset. Using metrics from the large multi-label classification dataset training, XceptionCov19 improves on COVID19XrayNet's performance by 0.0338, a 3.7\% increase in accuracy. COVID19XrayNet only trained on two databases that combined to a total of 6195 images however, even when XceptionCov19 trained on a smaller dataset of 1500 images it still produced a better result.

Comparing the results of XceptionCov19 against those of \cite{fitriasari2021improvement} Xception and ResNet inspired model, the multi-label classification results are closer. When \cite{fitriasari2021improvement} model trained on a total of 1398 images of 4 classes, compared to XceptionCov19's 3 multi-label classes, it produced accuracy results of 0.9341, a marginal improvement over XceptionCov19's 0.9335 when trained on the small dataset.

The \cite{bressem2020comparing} study compared many different models, much like this study, and its best performing model was DenseNet201. While the study does not provide accuracy results to compare with, 