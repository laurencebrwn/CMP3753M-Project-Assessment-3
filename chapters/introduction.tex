\section{COVID-19 and its Diagnosis}
Coronavirus (COVID-19) emerged early 2020 and has since spread around the world causing much disruption to daily living and claiming the lives of many. Several techniques to combat this disease have been developed, such as vaccination, isolation, and mass testing. While vaccination is having a positive impact in combating COVID-19, experts agree that mass testing is critical in defeating the spread of COVID-19 \citep{rosenthal2020importance}.

Two popular methods of diagnosing early are polymerase chain reaction (PCR) tests and lateral flow tests (LFT). These tests can be taken quickly, either at home or at a centre, and give a very accurate and reliable result, but are not 100\% accurate, with Area Under the Receiver Operating Characteristics Curve (AU-ROC) results of up to 0.879 for PCR tests \citep{mardani2020laboratory}. Once a patient is suspected positive for COVID-19 by one of the previous tests, this then needs to be confirmed and officially diagnosed by doctors using computerized tomography (CT) and X-Ray scans. This confirmation provides an accurate diagnosis of COVID-19, as well as giving doctors protection from misdiagnosis and medical negligence claims.

\section{Diagnosing COVID-19 Using Machine Learning}
Using machine learning to aid in the diagnosis process has several advantages. Firstly, as computers can work unsociable hours and are able to work more quickly, consistently and for longer periods than humans, it makes the process of sifting through large quantities of data easier for doctors. Secondly, computers can do this to a high degree of accuracy too, with computers generally being on par with a doctor’s success in diagnosis 72.5\% vs 71.4\% respectively in a clinical vignettes test \citep{richens2020improving}. Finally, computer predicted diagnoses can also be used as a second opinion for a doctor performing the diagnoses. 

This project aims to aid the implementation of this process by evaluating the performance of six Convolutional Neural Networks (CNN) when identifying COVID-19, through classification of X-Ray scans, and measuring the effectiveness of each to come to a decision as to which is most suitable for this application. Using hyper-parameter tuning and network modification, this project will improve upon the best networks to also recommend the best network configuration for the leading model. This decision can then be considered when developing a system to put into production to help doctors in this diagnoses process.